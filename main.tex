\documentclass[oneside,12pt,fleqn]{memoir}

\usepackage{makeidx}
\usepackage[columns=1]{idxlayout}
%\usepackage[utf8]{inputenc}
\pagestyle{plain}
\binoppenalty2000 \relpenalty2000

%%%%%%%%%%%%%%%%%%%%%%%%%%%%%%%%%%% importa pacchetti
\usepackage{usepkg}
%%%%%%%%%%%%%%%%%%%%%%%%%%%%%%%%%%%%% fancyhdr
\usepackage{fancyfoot}
%%%%%%%%%%%%%%%%%% titletoc, titlesec setting
\usepackage{titleT}
%%%%%%%%%%%%%%%%%% setlength
\usepackage{mylength}
\linespread{0.5}

%%%%%%%%%%%% Hyperref package
\usepackage{hyperref}
\hypersetup{colorlinks,
linktoc=all,
linkcolor=black,
    citecolor=black,
    filecolor=black,
    urlcolor=black
}
%%%%%%%%%%%%%%%%%%%%%%%%

%%%%%%%%%%%%%%%%%Geometry package
\usepackage{mygeometry}

%%%%%%%%%%%%%%%%%%%%%%%%%%%%%%%%%%% Funzioni per questo file main
\usepackage{LocalF}
%%%%%%%%%%%%%%%%%%%%%%%%%%%%%%%%%%% Funzioni generali
\usepackage{functions}
%http://tex.stackexchange.com/questions/246/when-should-i-use-input-vs-include
\usepackage{sources}
\usepackage{MathOp}
%%%%%%%%%%%%%%%%%%%%%%%%%%%%%%%%%
%\usepackage{tikz/data}%%import table for tikz pgfplot


\makeindex
\raggedbottom %http://tex.stackexchange.com/questions/102084/annoying-paragraph-spacing-issue-with-memoir 


%%%%%%%%%%%%%%%%%% Import mypackages
%\usepackage{mytitletoc} %% Remeber some problems occurs if you put this after packages
%\usepackage{packages}   %%
%\usepackage{mygeometry} %%
%\usepackage{functions}  %%
%\usepackage{LocalF} %%
%\usepackage{MathOp} %%
%%%%%%%%%%%%%%%%%%% Sources
%\usepackage{sources}
%%%%%%%%%%%%



\author{Pippetta}
\title{Astro Phys 1}
\date{\today}
%\date{\currenttime}

\makeindex
\raggedbottom %http://tex.stackexchange.com/questions/102084/annoying-paragraph-spacing-issue-with-memoir

%\csname @addtoreset\endcsname{figure}{chapter}

\begin{document}

\frontmatter
\maketitle
\addtocontents{toc}{\protect\hypertarget{toc}{}}
\tableofcontents*

\mainmatter


\part{Costanti e grandezze fisiche e fattori di conversione (che \'e meglio ricordare)}
 

\subfile{usefulldescriptions}

\part{Strutture autogravitanti in equilibrio}
%http://jilawww.colorado.edu/~pja/stars02/

\subfile{mechanicalequilibrium}

\part{Trasporto radiativo. Stabilit\'a rispetto a perturbazioni dello stato interno. Trasporto convettivo.}

\subfile{transportstability}


\part{Stato della materia, equazione di stato, ionizzazione, (produzione di energia, opacit\'a)}

\subfile{stellarplasmamodel}

\part{Modello ed evoluzione stellare.}

\subfile{stellarmodels}

\part{Osservabili stellari. Classificazione}

\subfile{stellarobservables}

\part{Descrizione semi-quantitativa dell'universo.}

\subfile{universetc}

\part{Esempi di semplici problemi astrofisici.}

\chapter{Pulsazioni}
\PartialToc

\section{Ic}

classespulsating


\renewcommand{\listfigurename}{Elenco figure}
 
\renewcommand{\indexname}{Indice}

\listoffigures

\printindex


\end{document}
