\documentclass[../main.tex]{subfiles}
\begin{document}
\tpc{!ht}{
\node[circle,fill,inner sep=1pt,label=above:M] (M) at (0,0) {}; 
\draw (M)--++(30:1cm) node[circle,fill,inner sep=1pt,label=below:O] (O) {};
\draw[->] (O)--++(30:1.5cm) node[circle,fill,inner sep=1pt,label=above:m,yshift=1pt,xshift=1pt] (m) {} node[midway,above] {$\vec{r}$} ;

\node[circle,fill,inner sep=1pt,below=1cm of O,label=below:O] (O1) {}; 
\draw[->] (O1)--++(30:1.5cm) node[circle,fill,inner sep=1pt,label=above:m,yshift=1pt,xshift=1pt] (m1) {} node[midway,below] {$\vec{r_1}$} ;
\draw[->] (O1)--++(-150:1cm) node[circle,fill,inner sep=1pt,label=above:m,yshift=1pt,xshift=1pt] (m1) {} node[midway,below] {$\vec{r_2}$} ;
\node (dida) at (7,0) {\parbox{8cm}{Siano m e M due masse puntiformi o a simmetria sferica: O \'e il centro di massa e $\vec{r}=\vec{r_1}-\vec{r_2}$ la distanza relativa.}};
}{(5,-2)}{10cm}{Problema ridotto}
\end{document}
